%        File: drift_effects.tex
%     Created: Sun Aug 02 11:00 AM 2009 C
% Last Change: Sun Aug 02 11:00 AM 2009 C
%
\documentclass[letterpaper,10pt]{article} 
\usepackage[pdftex]{color,graphicx} 
\usepackage{amsmath, amsfonts, amssymb, latexsym, inputenc, moreverb, wrapfig, subfigure, array, lscape, setspace, cite}
\usepackage[pdftex, colorlinks]{hyperref}
\newcommand{\ud}{\mathrm{d}} 
\newcommand{\degree}{$^{\circ}$}
\newcommand{\tb}[1]{\textcolor{blue}{#1}} 
\newcommand{\E}{\mathbb{E}}

%\usepackage{fullpage} %% 1 inch margins by default
%\usepackage{pslatex}  %% use normal postscript fonts (like times new roman)

\title{}
\author{Carl Boettiger}
\begin{document}
\maketitle

\begin{align*}
\ud n_1 = r n_1 \left(1 -  \frac{n_1 + C(x_1, x_2) n_2}{K(x_1) } \right) \ud t + \frac{1}{\sqrt{K_o} } \sqrt{r n_1 \left(1 +  \frac{n_1 + C(x_1, x_2) n_2}{K(x_1) } \right) } \ud W_1
\end{align*}

\begin{align*}
p &= \frac{n_1}{n}, \quad q = 1-p = \frac{n_2}{n} \quad n = n_1 + n_2 \\
\ud p &= \frac{\ud n_1}{n}, \quad \ud q = \frac{\ud n_2}{n} \\
\ud n &= \ud n_1 + \ud n_2 = n (\ud p + \ud q) 
\end{align*}

\begin{align*}
\ud p(p,n) = \alpha_1(p,n) \ud t + \sqrt{ \alpha_2(p,n) } \ud W_p \\
\ud n(p,n) = \beta_1(p,n) \ud t +  \sqrt{ \beta_2(p,n) } \ud W_n
\end{align*}

\begin{align*}
f_i(p) =  r p \left(1 -  n(p) \frac{p + c_{12}(1-p) }{k_i } \right) \\
g_i(p) =  r p \left(1 +  n(p) \frac{p + c_{12}(1-p) }{k_i } \right) 
\end{align*}

\begin{align*}
\alpha_1 &=  f_1(p) \\ 
\alpha_2 &= \frac{ g_1(p) }{K_0 n(p) } \\
\beta_1 &= n f_1(p) + n f_2(1-p)  \\
\beta_2 &= n*\frac{g_1(p)+g_2(1-p)}{K_0 } 
\end{align*}


\begin{align*}
\frac{\ud}{\ud t} \sigma_p &= 2 \partial_p \alpha_1(p,n) \sigma_p  +2\partial_n\alpha_1(p,n) \langle p n \rangle +  \alpha_2(p,n) \\
\frac{\ud }{\ud t} \sigma_n&= 2 \partial_n \beta_1(p,n) \sigma_n  +2\partial_p\beta_1(p,n) \langle p n \rangle +  \beta_2(p,n) \\
\frac{\ud }{\ud t} \langle p n \rangle &= \partial_n \alpha_1(p,n) \sigma_n  +\partial_p \alpha_1 \langle pn \rangle + \partial_p\beta_1 (p,n) \sigma_p + \partial_n \beta_1(p,n) \langle p n \rangle
\end{align*}


\begin{align*}
\partial_p \alpha_1 &= 
\\ \partial_n \alpha_1 &=
\\ \partial_p \beta_1 &= 
\\ \partial_n \beta_1 &=
\end{align*}


\end{document}


