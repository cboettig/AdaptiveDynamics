\documentclass[authoryear,1p,10pt]{elsarticle}
\bibliographystyle{elsarticle-harv}
%\usepackage{graphicx}
\usepackage{amsmath, amsfonts, amssymb, latexsym, wrapfig, subfigure, array, setspace, cite}
\usepackage{lineno}
%\linenumbers
%\usepackage[pdftex]{color}
%\definecolor{darkblue}{rgb}{0,0,0.5}
%\definecolor{darkgreen}{rgb}{0,0.5,0}
%\usepackage[pdftex, colorlinks, citecolor=darkblue,linkcolor=darkgreen]{hyperref}
%\usepackage[pdftex, colorlinks]{hyperref}
\textwidth 6.75in
\oddsidemargin -0.15in
\evensidemargin -0.15in
\textheight 9in
\topmargin -0.5in
\newcommand{\ud}{\mathrm{d}}
\newcommand{\degree}{$^{\circ}$}
\newcommand{\tb}[1]{\textcolor{blue}{#1}} 
\newcommand{\E}{\text{E}}



 
\begin{document}
\begin{frontmatter}
\title{Stochastic Impediments to Evolutionary Branching}
\author[carl]{Carl Boettiger\corref{cor1}}
\ead{cboettig@ucdavis.edu}
\cortext[cor1]{Corresponding author.}
\author[rupert]{Rupert Mazzucco\corref{cor1}}
\ead{mazzucco@iiasa.ac.at}
\author[ulf]{Ulf Dieckmann\corref{cor1}}
\ead{dieckmann@iiasa.ac.at}
\address[carl]{Center for Population Biology, University of California, Davis, United States}
\address[rupert]{International Institute for Applied Systems Analysis, Austria}
\address[ulf]{International Institute for Applied Systems Analysis, Austria}
\begin{abstract}
We show stochastic effects due to finite population sizes can pose a significant impediment to evolutionary branching.   We investigate these effects by exploring the waiting time until evolutionary branching occurs using both individual-based simulationand analytic approximations.  The accuracy of our approximation demonstrates that adaptive branching can be thought of as occurring in four phases: (1) (2) (3) (4).  Different ranges of parameters will make different phases become rate-limiting.  We find that the delicate balance of coexistence early in evolutionary branching is most often rate-limiting, and provide a convenient approximation to the waiting time based on this limit.   
\end{abstract}
\begin{keyword}
Evolution \sep demographic stochasticity \sep branching \sep adaptive dynamics
\end{keyword}
\end{frontmatter}
\section{Introduction}
\begin{itemize}
\item Sympatric speciation and the adaptive dynamics of evolutionary branching
\item Previous work on stochasticity in branching \citet{Claessen2007a, Claessen2007, Claessen2008, Johansson2006}
\item Summarize results and outline paper 
\end{itemize}
\section{Theory and methods}
\subsection{Model Formulation}
Paragraph reviewing basics of evolutionary branching, \citet{Geritz1997}, \citet{Dieckmann1999a}.
\subsubsection{Rosenzweig model of competition for a limiting resource}
Paragraph reviewing the competition model
\subsubsection{Individual-based simulation}
Paragraph reviewing the individual based model implementation
\subsection{Four phases of evolutionary branching}
\begin{enumerate}
\item Convergence the branching point
\item Invasion of the coexistence region
\item Coexistence until next invasion
\item Divergence from the branching point.
\end{enumerate}
Figure 1: with two panels:  (a) shows eachs of these phases on the Pairwise Invasibility Plot. (b) Histograms for each showing the absolute population abundance at each trait value during each of the phases.   

\section{Results}
\subsection{Full Approximation}
Figure 2: Distribution of waiting times from simulation, with full approximation fit, with rate-limiting approximation from phase 3.
\subsection{Rate-limiting coexistence until next invasion}
Figure 3: Escape from potential energy well approach used to calculate the coexistence time
\subsection{Other rate-limiting steps}
Figure 4: Conceptual figure showing biological scenario corresponding to each limit, the resultant approximation, and simulation from that limiting case.


\section{Discussion}
\begin{itemize}
\item What understanding waiting times tells us about the evolutionary branching process
\item How quantifying rate limiting steps helps identify the most relevant parameters to measure in determining rates of evolutionary branching.  
\item Extensions of the approach, such as the inclusion of environmental variation
\end{itemize}

\section{Acknowledgements}
\begin{itemize}
\item IIASA
\item NAS, DoE funding
\item other?
\end{itemize}
\bibliography{AdaptiveDynamics.bib}
\end{document}


