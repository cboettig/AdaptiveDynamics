\begin{frontmatter}
\title{Waiting for Speciation: Estimating the time until Evolutionary Branching Occurs}
\author[carl]{Carl Boettiger\corref{cor1}}
\ead{cboettig@ucdavis.edu}
\cortext[cor1]{Corresponding author.}
\author[rupert]{Rupert Mazzucco\corref{cor1}}
\ead{mazzucco@iiasa.ac.at}
\author[ulf]{Ulf Dieckmann\corref{cor1}}
\ead{dieckmann@iiasa.ac.at}
\address[carl]{Center for Population Biology, University of California, Davis, United States}
\address[rupert]{International Institute for Applied Systems Analysis, Austria}
\address[ulf]{International Institute for Applied Systems Analysis, Austria}
\begin{abstract}
The theory of Adaptive Dynamics has recieved much attention for its analysis of evolutionary branching points.  While the asymptotic dynamics of such branching points has been well studied, little is known about the timescale of the process or its behaviour away from the asymptotic limits of large populations and rare, small mutational steps.  We provide several estimates of the waiting time for evolutionary branching to occur when different assumptions are relaxed.  This approach demonstrates how adaptive branching can be characterized by four seperate phases and also identifies the most likely rate-limiting process in the course of adaptive branching.  Our analytic approximations are compared to exact numerical simulation of the branching process.  
\end{abstract}
\begin{keyword}
Evolution \sep demographic stochasticity \sep branching \sep adaptive dynamics
\end{keyword}
\end{frontmatter}

